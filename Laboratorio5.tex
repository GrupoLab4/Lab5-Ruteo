\documentclass{udpreport}
\documentclass{article}
\usepackage[export]{adjustbox}
\title{Ruteo estático y dinámico. }
\author{Integrantes: Thomas Muñoz, Ignacio Yanjari, Dagoberto Navarrete, Ignacio López.}
\date{17 de Julio de 2016}
\usepackage{graphicx}
\usepackage{float}
\graphicspath{ {images/} }
\udpschool{Escuela de Informática y Telecomunicaciones}

\begin{document}
\maketitle
\tableofcontents
\listoffigures
\chapter{Introducción}
  
\chapter{Actividades}
	\section{Topología base}
	Lo primero que se procedió a hacer es abrir el Cisco Packet Tracer, posteriormente se continua a sacar ocho ordenadores,
	cuatro switch y cuatro routers. Una vez con los dispositivos en la interfaz se continua a posicionarlos en el espacio de
	manera que quedaran como la topología mostrada, luego se conectaron los ordenadores a los switch y los switch a los
	routers, hecho esto se procede a instalar el modulo serial en los routers, por lo que eran apagados y luego se les insertaba
	el modulo. Hecho lo anterior se procede a conectar los router entre si esto se logra a través de un cable serial, una vez
	terminado esto se continua a seguir con las actividades posteriores.\\
	\section{Configuracion Equipos}
	\section{Configuracion ruteo estático}
	\section{Configuracion ruteo dinámico}
{\large \bf{Cuestionario: }}\\
	\begin{enumerate}
	    \item ¿Qué ventajas y desventajas se pueden apreciar en cada tipo de enrutamiento?\\\\
            \item ¿En que se basa el enrutamiento dinámico para generar su ruta?\\\\
 
  	     
	\end{enumerate}
	
    
	
\chapter{Conclusión}
 
\begin{thebibliography}{x}
\bibitem{Website Osi} \textsc{Cisco },
\textit{cisco.utmetropolitana.edu.mx}


\end{thebibliography}

\end{document}
