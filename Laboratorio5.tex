\documentclass{udpreport}
\documentclass{article}
\usepackage[export]{adjustbox}
\title{Ruteo estático y dinámico. }
\author{Integrantes: Thomas Muñoz, Ignacio Yanjari, Dagoberto Navarrete, Ignacio López.}
\date{17 de Julio de 2016}
\usepackage{graphicx}
\usepackage{float}
\graphicspath{ {Imagenes/} }
\udpschool{Escuela de Informática y Telecomunicaciones}

\begin{document}
\maketitle
\tableofcontents
\listoffigures
\chapter{Introducción}
  En este laboratorio tenemos como objetivo aprender a hacer funcionar de forma correcta una red para que las subredes de esta se
  pueden comunicar entre ellas, entender en que consiste el ruteo estático y el ruteo dinámico, cuáles son sus diferencias y las
  ventajas y desventajas de cada uno, también aprenderemos como configurar una red con cada tipo de ruteo y dentro del ruteo dinámico
  veremos lo que son los protocolos de enrutamiento, enfocándonos en RIP (Routing Information Protocol).
\chapter{Actividades}
	\section{Topología base}
	Lo primero que se procedió a hacer es abrir el Cisco Packet Tracer, posteriormente se continua a sacar ocho ordenadores,
	cuatro switch y cuatro routers. Una vez con los dispositivos en la interfaz se continua a posicionarlos en el espacio de
	manera que quedaran como la topología mostrada, luego se conectaron los ordenadores a los switch y los switch a los
	routers, hecho esto se procede a instalar el modulo serial en los routers, por lo que eran apagados y luego se les insertaba
	el modulo. Hecho lo anterior se procede a conectar los router entre si esto se logra a través de un cable serial, una vez
	terminado esto se continua a seguir con las actividades posteriores.\\
	\begin{figure}[H]
	\centering
	\includegraphics[width=\textwidth]{Topologia_base.PNG}
	\caption{Construcción de la Topología Base}
	\end{figure}
	\newpage
	\section{Configuración Equipos}
	Se configuran los equipos de manera que cada PC pertenezca a una red distinta, esto se hace en las configuraciones de cada PC
	donde se le asignan IPs manualmente, y luego se configura el router asignándole una IP perteneciente a la red común con los 
	PC.\\
	\begin{figure}[H]
	\centering
	\includegraphics[width=\textwidth]{Configuracion_equipos.PNG}
	\caption{Configuracion de los equipos}
	\end{figure}
	\section{Configuración ruteo estático}
	Se ingresa a la terminal del router, luego indicándoles la IP de destino, porque puerto sale, y la IP por donde da el salto
	para llegar a destino, esto se debe hacer con cada interfaz del router, de manera que dirija cada paquete que reciba a una
	interfaz y lo pueda enviar. Lo anterior descrito se debe hacer con cada router perteneciente a la topología, para que así las
	redes queden todas conectadas entre sí.\\
	\begin{figure}[H]
	\centering
	\includegraphics[width=\textwidth]{Ruteo_estatico.PNG}
	\caption{Configuracion de ruteo estatico en los router R0 y R3}
	\end{figure}
	\section{Configuracion ruteo dinámico}
	Se ingresa a la terminal del router y se activa RIP V1 de esta manera se configura automáticamente y solo tenemos que
	indicarle al router cuál es la red que va a utilizar. Esto se logra al ingresar los comandos que estaban especificados en la
	guía.\\
	\begin{figure}[H]
	\centering
	\includegraphics[width=\textwidth]{Ruteo_dinamico.PNG}
	\caption{Configuracion de ruteo dinamico en los router R0 y R3}
	\end{figure}
{\large \bf{Cuestionario: }}
	\begin{enumerate}
	    \item ¿Qué ventajas y desventajas se pueden apreciar en cada tipo de enrutamiento?\\\\
	    \underline {VENTAJAS}:\\
	    Enrutamiento Estático:
	    \begin{itemize}
	    	\item El consumo de recursos del router es bajo.
	    	\item Es fácil de configurar en redes pequeñas.
	    \end{itemize}
	    Enrutamiento Dinámico:
	    \begin{itemize}
	    	\item En el mantenimiento de las rutas el encargado casi no debe participar.
	    	\item Esta configuración tiende a tener menos errores ya que los routers hacen el trabajo de actualización de la red periodicamente.
	    	\item Las tablas de ruteo se calculan mediante un algoritmo decidiendo cual es la ruta óptima que debiese tomar un paquete.
	    \end{itemize}
	    \underline{DESVENTAJAS}:\\
	    Enrutamiento Estático:
	    \begin{itemize}
	    	\item Para redes grandes es complicado de configurar.
	    	\item La configuración es propensa a errores ya que es el administrador quien ingresa las rutas manualmente.
	    	\item La configuración, mantención y actualización de rutas se debe hacer manualmente.
	    	\item No permite una escalabilidad en caso de gran crecimiento de la topología.
	    \end{itemize}
	    Enrutamiento Dinámico:
	    \begin{itemize}
	    	\item El consumo de recursos del router es alto.
	    	\item Se requiere un conocimiento avanzado.
	    %(SIEMPRE SE PUEDEN AGREGAR MÁS)
	    \end{itemize}
            \item ¿En que se basa el enrutamiento dinámico para generar su ruta?\\
            	Se basa en la comunicación entre todos los routers de la red, los cuales comparten sus tablas de ruteo indicando en un principio cuales son las redes que están conectadas directamente a sus interfaces y periódicamente comparten sus tablas para mantener toda la red actualizada en caso de alguna modificación.
            	
            	El funcionamiento es el siguiente: Cada router envía un mensaje por todas sus interfaces pidiéndole a los otros que le envíen sus tablas de ruteo. Luego cada cierto tiempo los routers difunden mensajes indicando cuales son las redes a las que acceden sus interfaces y la distancia a la que están de ellas.
 
  	     
	\end{enumerate}
	
    
	
\chapter{Conclusión}
 Al final de esta experiencia hemos aprendido cuales son las características principales de cada método de enrutamiento y junto con
 esto los aspectos a evaluar para saber cuál de los dos métodos debemos usar en los routers de la red en cuestión, y a configurar
 ambos, tomando en cuenta los protocolos de enrutamiento dinámico en caso de elegir ese método.\\
\begin{thebibliography}{x}
\bibitem{Website Osi} \textsc{Cisco },
\textit{cisco.utmetropolitana.edu.mx}
\textit{es.slideshare.net/eduardoelange/diferencias-entre-enrutamiento-esttico-y-dinmico}

\end{thebibliography}

\end{document}
